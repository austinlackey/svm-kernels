\documentclass[12pt]{article}
\usepackage{hyperref}
\usepackage{authblk}
\usepackage{graphicx}
\usepackage{amsmath}
\usepackage{amssymb}
\usepackage{enumitem}
\usepackage{amsfonts}
\usepackage{array}
\usepackage{xcolor}
\usepackage{tikz}
\usepackage{pifont}
\usepackage{fontawesome5}
\usepackage{tikz-3dplot}
\usepackage{forest}
\usepackage{cancel}
\usepackage{ulem}
\usepackage[margin=1in]{geometry}

\title{Methodology of Creating SVM Kernels from Scratch Using Python and NumPy}
\author{Austin Lackey}
\author{Tomy Sabalo Farias}
\affil{DSCI 320, Colorado State University}

\begin{document}

\maketitle

\begin{abstract}
This paper presents a methodology for creating Support Vector Machine (SVM) kernels from scratch using Python and NumPy. We discuss the implementation of linear, sigmoid, polynomial, and radial basis function (RBF) kernels in a binary SVM and a multiclass SVM.
\end{abstract}

\section{Introduction}
In this section, provide an introduction to SVMs, their applications, and the importance of kernels in SVMs.

\section{Methodology}
In this section, describe the methodology used to create the SVM kernels from scratch.

\subsection{Binary SVM}
Discuss the implementation of the Binary SVM class, including the implementation of the different kernels.

\subsection{Multiclass SVM}
Discuss the implementation of the Multiclass SVM class, which uses the Binary SVM class.

\section{Results and Discussion}
In this section, present and discuss the results obtained using the implemented SVM kernels.

\section{Conclusion}
In this section, provide a conclusion summarizing the work done and its implications.

% THESE ARE EXAMPLES OF HOW TO USE BIBTEX
\begin{thebibliography}{9}
\bibitem{numpy}
Travis E, Oliphant. 
A guide to NumPy, USA: Trelgol Publishing, (2006).

\bibitem{python}
Van Rossum, G., Drake, F.L. 
Python 3 Reference Manual, Scotts Valley, CA: CreateSpace, (2009).

\bibitem{svm}
Cortes, C., Vapnik, V. 
Support-vector networks. Machine Learning, 20(3):273-297, (1995).
\end{thebibliography}

\end{document}