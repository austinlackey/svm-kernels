\documentclass[12pt]{article}
\usepackage{hyperref}
\usepackage{authblk}
\usepackage{graphicx}
\usepackage{amsmath}
\usepackage{amssymb}
\usepackage{enumitem}
\usepackage{amsfonts}
\usepackage{array}
\usepackage{xcolor}
\usepackage{tikz}
\usepackage{pifont}
\usepackage{fontawesome5}
\usepackage{tikz-3dplot}
\usepackage{forest}
\usepackage{cancel}
\usepackage{ulem}
\usepackage{booktabs}
\usepackage[margin=1in]{geometry}

\title{Methodology of Creating SVM Kernels from Scratch Using Python and NumPy}
\author{Austin Lackey}
\author{Tomy Sabalo Farias}
\affil{DSCI 320, Colorado State University}

\begin{document}

\maketitle

\begin{abstract}
This paper presents a methodology for creating Support Vector Machine (SVM) kernels from scratch using Python and NumPy. 
We discuss the implementation of linear, sigmoid, polynomial, and radial basis function (RBF) kernels in a binary and multiclass SVM.
These kernels are tested on E. coli data and compared to the results of the scikit-learn SVM implementation.
The results show that the implemented kernels often yield better accuracy than the scikit-learn implementation; at the
cost of increased training time. Kernel training times are ran multiple times and averaged to provide a more accurate
metric since training times are low and have a high variance.
\end{abstract}

\section{Introduction}
In this section, provide an introduction to SVMs, their applications, and the importance of kernels in SVMs.

\section{Methodology}
In this section, describe the methodology used to create the SVM kernels from scratch.
\subsection{Kernel Functions}
We implemented four different kernel functions which are listed below.
\begin{itemize}
    \item Linear Kernel
    \begin{equation}
    K(X, Y) = X^T Y
    \end{equation}
    
    \item Sigmoid Kernel
    \begin{equation}
    K(X, Y) = \tanh(\gamma X^T Y + r)
    \end{equation}
    
    \item Polynomial Kernel
    \begin{equation}
    K(X, Y) = (\gamma X^T Y + r)^d, \gamma > 0
    \end{equation}
    
    \item Radial Basis Function (RBF) Kernel
    \begin{equation}
    K(X, Y) = \exp(-\gamma ||X - Y||^2), \gamma > 0
    \end{equation}
\end{itemize}
\subsection{Binary SVM}
Discuss the implementation of the Binary SVM class, including the implementation of the different kernels.

\subsection{Multiclass SVM}
Discuss the implementation of the Multiclass SVM class, which uses the Binary SVM class.

\section{Results and Discussion}
In this section, present and discuss the results obtained using the implemented SVM kernels.
\begin{table}[h]
    \centering
    \caption{Classifier Performance}
    \label{tab:classifier_performance}
    \begin{tabular}{cccccc}
        \toprule
        \textbf{Classifier} & \textbf{Implementation} & \textbf{Kernel} & \textbf{Avg Accuracy} & \textbf{Avg Runtime} \\
        \midrule
        Binary & sklearn & linear & 1.00000 & 0.00052 \\
        Binary & custom & linear & 0.99318 & 0.00795 \\
        Binary & sklearn & sigmoid & 0.68180 & 0.00066 \\
        Binary & custom & sigmoid & 0.99546 & 0.00757 \\
        Binary & sklearn & rbf & 1.00000 & 0.00054 \\
        Binary & custom & rbf & 0.99546 & 0.00977 \\
        Binary & sklearn & poly & 1.00000 & 0.00040 \\
        Binary & custom & poly & 0.97498 & 0.00929 \\
        Multi & sklearn & linear & 0.77940 & 0.00096 \\
        Multi & custom & linear & 0.81468 & 0.15420 \\
        Multi & sklearn & sigmoid & 0.47060 & 0.00183 \\
        Multi & custom & sigmoid & 0.82497 & 0.15641 \\
        Multi & sklearn & rbf & 0.75000 & 0.00156 \\
        Multi & custom & rbf & 0.82350 & 0.19177 \\
        Multi & sklearn & poly & 0.76470 & 0.00111 \\
        Multi & custom & poly & 0.83967 & 0.36326 \\
        \bottomrule
    \end{tabular}
    \vspace{1em}  % Add some vertical space for the note
    \begin{tabular}{p{0.9\textwidth}}  % Create a new table for the note
        \multicolumn{1}{l}{\textit{Note:}} These times were calculated with 10 runs for each kernel and averaged. \\
    \end{tabular}
\end{table}
\section{Conclusion}
In this section, provide a conclusion summarizing the work done and its implications.

% THESE ARE EXAMPLES OF HOW TO USE BIBTEX
% \begin{thebibliography}{9}
% \bibitem{numpy}
% Travis E, Oliphant. 
% A guide to NumPy, USA: Trelgol Publishing, (2006).

% \bibitem{python}
% Van Rossum, G., Drake, F.L. 
% Python 3 Reference Manual, Scotts Valley, CA: CreateSpace, (2009).

% \bibitem{svm}
% Cortes, C., Vapnik, V. 
% Support-vector networks. Machine Learning, 20(3):273-297, (1995).
% \end{thebibliography}

\end{document}